%%%PRACTICAL THEORY
\noindent
We open the discussion with a quote \citep{Lewin1943}:
\begin{displayquote}
... there is nothing as practical as a good theory. (p. 118).
\end{displayquote}
The rational theory proved extremely practical, in the scientific sense.  Given impoverished naturalistic observational data, the theory afforded a mathematical decision modeling framework grounded in prior experimental work\citep{GriffithsTenenbaum2006,GriffithsTenenbaum2011}, a set of intriguing (if not paradoxical) predictions (used in Study 1) and a potential limitation (use of external information, Study 2).  

The application of the model to the naturalistic observational data returned the favor. First, it offers fresh testable hypotheses.  In Study 1, we put forth the notion that participants were sensitive to and reacted to the difference between the $t_{predicted}$ and the prior; there was some evidence for what seemed like a dynamic response once this difference became relatively large.  This amounts to a testable, first-order hypothesis: are people actually sensitive to this measure?  To speculate, it should be possible to develop an experimental protocol to measure this hypothetical phenomenon that leverages existing experimental procedures and measures, e.g., \citep{sussman2007role}.  In Study 2, we saw a striking relation between the human predictions and the case-counts, one that is suggestive of an understanding of the epidemiological process on the part of the participants. 

Given said hypotheses, the model provides a second service--pointers for the construction of new theoretical processes.  The scope of this article does not warrant proposing a new or amended model to accommodate external information (Study 2) or to incorporate the difference between $t_{predicted}$ and the prior (Study 1).  However, the flexibility of the theoretical approach is very general as demonstrated by its ability to accommodate a variety of findings even within narrow domains, e.g., anchoring bias \citep{Lieder2018}. 

The rational theory also proved practical, in the practical sense.  This work represents a step towards improving the psychological underpinnings of the kinds of human judgements that are directly relevant for a variety of policy, administrative or operational needs\citep{Galesic2021}.  The tournament from which we gathered our data was developed in conjunction with the Virginia Department of Health (VDH) to provide insights into future case numbers, vaccination rates, booster uptake and other key indicators for VDH COVID-19 operations.  Our work represents only a slice of the tournament, but it begins the process of providing a scientific evidence-base for the forecasts, something valued in near real-time emergency decision-making \citep{Galea2021}.  In this policy space, improved human-machine collaborations (especially when theory of mind is desirable) may yield benefit.  

Another relevant application of our work in the policy decision space is towards developing high-fidelity population models of infectious disease.  Our work could provide the basis for the decision logic used by synthetic agents in large, at-scale agent-based models of infectious disease dynamics, an effort for which our laboratory is highly-experience (University of Virginia).  Recent work has called for the integration of agents grounded in first-principles of cognitive science and psychology \citep{orrAttPolar2021,afrasiabi2019evaluating,orr2019multi,bhattacharya2019matrix,orrCovid19}.

In sum, the benefit of intermingling psychological theory with naturalistic data is clear from our work.  More varieties of this kind will benefit both psychological science and society.  