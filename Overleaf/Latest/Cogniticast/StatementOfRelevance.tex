%%%must include a 150-word Statement of Relevance that explains why the research reported in the submission is of interest and significance beyond the specific sub-area in which it is situated and, ideally, to the public at large. The Statement of Relevance does not count toward the word limit. The aim of the Statement of Relevance is to broaden the impact of the science reported in the journal and make it easier for interested readers to appreciate and understand our efforts. It should make clear why the questions that motivated the study and the findings that bear on them matter beyond psychology laboratories and college and university campuses. What is requested is a description of the sort that might open a conversation with a journalist, explain the work to a friend or family member, or introduce a student to the field of inquiry. In other words, a Statement of Relevance is not a technical abstract but instead, a description that makes the work accessible beyond the professional academe.

Human predictions of within-event duration (prediction of how much longer a process or event will last) play an important role in everyday life (planning) and extraordinary life (pandemic preparedness for governments).  The scientific understanding of predictions of event durations is hindered by a lack of connection between experimental work, theoretical mathematical models and real-time, naturalistic observation in which event duration is judged repeatedly over time by the same set of participants.  In this study, we observed conditions in a naturalistic forecasting tournament that address unrecognized issues for the prevailing Bayesian model of event duration.  In particular, under conditions like our naturalistic context, the Bayesian model can infer somewhat drastic changes priors used in the decision process.  Such drastic changes, which could be argued to be excessively large, are avoidable given adaptation in the participant response.  Our analysis provides provisional and qualified evidence that adaptation was present for some participants.  This study, in addition to pushing the theoretical concerns further and providing novel, testable hypotheses, shows a useful case at the intersection of (i) applied, real-world settings and (ii) experimentally-derived mathematical theory of behavior.  Because our data come from an operational system (used by state government) a rigorous scientific underpinning of the psychological process is extra critical. 

