%\documentclass{article}
%\usepackage[utf8]{inputenc}
%\usepackage{graphicx}
%\graphicspath{ {./Figures/} }


\documentclass[a4paper,man,natbib]{apa6}

\usepackage{subcaption}

\usepackage[english]{babel}
\usepackage[utf8x]{inputenc}
\usepackage{amsmath}
\usepackage{mathtools}
\usepackage{graphicx}
\usepackage[colorinlistoftodos]{todonotes}
\usepackage{comment}
\usepackage{csquotes}
%%FOR TABLE THREE COMMANDS
\setlength{\arrayrulewidth}{0.3mm}
\setlength{\tabcolsep}{10pt}
\renewcommand{\arraystretch}{1.2}
\usepackage{outlines}
\usepackage{enumitem}
\setenumerate[1]{label=\Roman*.}
\setenumerate[2]{label=\Alph*.}
\setenumerate[3]{label=\roman*.}
\setenumerate[4]{label=\alph*.}
\usepackage{todonotes}
%%ADDED
\usepackage[font=footnotesize,labelfont=bf]{caption}

\title{On Optimality and Human Prediction of Event Duration in Real-Time, Real-World Contexts}
\shorttitle{Real-Time Optimal Event Duration}
\author{Biocomplexity Institute; Metaculus; VDH}
\affiliation{TBA}

\abstract{Your abstract here.}

\begin{document}
\maketitle


\section{Introduction}\label{introduction}
Human judgments of future events, what we will call forecasts or predictions, are ubiquitous and span the mundane (e.g., will the car in front of me stop suddenly) to the extraordinary--e.g., decisions made during military operation. The scientific study of human forecasting, fittingly, draws upon myriad domains and literature. The focus of the current report is what is called prediction of event duration, a specific domain that is defined by forecasting how much longer an event will continue given the point in time at which the prediction is called.  It is similar to being asked: "We've been at war for 2 months; how much longer will the war last?"
 
In this article, we explore event duration prediction in the case of real-world epidemics.    Specifically, the setting of the current work is the COVID-19 Omicron wave. We observed two rounds of an online forecasting tournament focused on characteristics of the COVID-19 epidemic in Virginia from late 2021 to early 2022.  We focused on forecasts of event duration, specifically the timing of the Omicron wave.   

The nature of these data, from a psychological perspective, are impoverished.  There was little to work with beyond the prediction itself.  Because of this, our methodological notion, born from adaptive, rational analysis \citep{anderson2013adaptive, Lieder2020, marr2010vision}, was that a useful first step in understanding the psychological processes and representations is to examine the problem from the outside (see \cite{simon1969sciences} for early exposition on this idea).  Simply put--to figure out how something works, first understand the/an optimal way in which it does what it does (see \cite{Tauber2017} for detailed exposition on this notion and related alternatives).  

Rational models have shown promise in behavioral tasks related to event duration prediction:  (i) humans can estimate the an event's duration in an optimal way given little evidence and (ii) prior knowledge is combined with new, observed data in an optimal way \citep{GriffithsTenenbaum2006,GriffithsTenenbaum2011,Tauber2017}.  (These results come with some degree of theoretical contention, e.g., \cite{MozerPashlerHomaei2008}).  The basic model of this kind of decision process, the one we adapted in our methods, is a Bayesian decision model derived from: 
\begin{equation}\label{eq:posteriorfori}
    P(t_{total} \vert t_{past}) = \frac{P(t_{past} \vert t_{total}) P(t_{total})} {P(t_{past})}
\end{equation}
where $t_{total}$ is a particular value of an event's \textit{total} duration, coming from a distribution of durations; $t_{past}$ is the duration at which a decision is made, or if post-hoc, the point for which the decision is made.  A simple example should clarify these terms:  

\textbf{[CUT THIS AND NEXT PARAGRAPH SO JUST EXPLAINS AT HIGHER LEVEL.]}

A person, Pat, is traveling on a train on a familiar route. After 20 minutes from departure, Pat is asked by another passenger how long will the train trip last (i.e., what will be the total duration in minutes).  For this example, $t_{past}$ is 20 minutes and $t_{total}$ is one value from a distribution encoded in memory by Pat's past experience on this train route, e.g., 45 minutes. Pat's response, according to the Bayesian decision model, is formulated as the expected value of the probability distribution generated from Equation \ref{eq:posteriorfori} over the distribution of all $t_{total}$. We call this decision $t_{predicted}$ to capture the notion that the rational model predicts a value of $t$ to be the duration of the current train trip.  In Bayesian terms, $P(t_{total})$, over all instances of $t_{total}$, defines the prior distribution; $P(t_{past} \vert t_{total})$ is the likelihood; $P(t_{total} \vert t_{past})$, over all instances in the prior, is the posterior distribution for a single value of $P(t_{past})$.

We can define a Bayesian prediction function (borrowing from \cite{GriffithsTenenbaum2006}) as $t_{predicted}$ over all values of $t_{past}$.  Prior results have shown that, with a Gaussian prior, the Bayesian prediction function is nonlinear: given values of $t_{past}$  $<<$ the mean, the approximate value of $t_{predicted}$ is the mean; however, as $t_{past}$ approaches the mean value of the prior, the value of $t_{predicted}$ increases to values larger than the mean value of the prior.  Once $t_{past}$ is past the mean it converges to $t_{predicted}$ slowly (see Figure 1 of \cite{GriffithsTenenbaum2006}).  This prediction function accords with human predictions under experimental conditions \citep{GriffithsTenenbaum2006} in which human predictions are provided in reference to a single event (e.g., ask participants to judge how long a train ride will last if a person has been sitting on a train for 20 minutes).

Experimental evidence supports the hypothesis that, for event duration prediction, the Bayesian decision model is affected by both prior, durable knowledge and new observations by the integration of the prior $P(t_{total})$ and the likelihood function $P(t_{past} \vert t_{total})$, respectively \citep{GriffithsTenenbaum2011}.  In this work, the experimental manipulation of the prior was based on either verbal scenario manipulation (e.g., describe verbally contextual information related to the prior about the expected duration) or through direct experience in the laboratory. For the latter, participants were provided with direct, controlled experience with relevant materials (see \cite{GriffithsTenenbaum2011} Experiment 4).  In terms of the model, this amounts to manipulation of $P(t_{total})$ through direct experience.   

The integration of new observations, in terms of the decision model, operates on the likelihood function defined as:
\begin{equation}\label{eq:likelihoodN}
   \left(\frac{1}{t_{total}} \right)^n
\end{equation}
(when $t_{total} \ge t_{past}$) where $n$ is the number of observations (see \cite{GriffithsTenenbaum2011} p. 729 for derivation).  Thus, as the number of observations increases, the predicted $t_{predicted}$ decreases, assuming the prior is invariant.  This is the assumed case for statistically independent observations.  However, the experimental evidence suggested that for dependent observations, $n$ is effectively equal to one, suggesting provisionally that humans are sensitive to the generating process of new observations.  Independence of observations is required for a change in the likelihood.

%%%
\subsection{Real-Time, Real-World Application of the Model}
The data we use for the current article capture an interesting real-time, real world context.  In the forecasting tournament that generated our data, participants were allowed to make repeated predictions over time, during a single event (in this case the event was the Omicron wave of COVID-19).  This is akin to asking Pat (from the example above) to predict the length of a single train trip repeatedly over the course of the trip at different values of $t_{past}$. This observational context--uncontrolled repeated observations in real-time--is useful because it affords a unique, applied context for understanding the implications of the rational decision model described above.  

Two issues arise when considering this observational context.  First, its structure has the peculiar potentiality of eliciting $t_{predicted}$ as $t_{past}$ approaches $t_{predicted}$.  Reflecting on our hypothetical train trip, Pat can make predictions up to the minute the train trip ends and Pat can make predictions that are wrong, e.g. predicting that the train trip ends now when it, instead, continues on. 

From the perspective of the rational decision model, this context brings to the fore a seemingly paradoxical conclusion of the rational model, one that is inherent in its construction and best shown by illustration.  Let us continue on our journey with Pat.  Imagine that Pat is riding on a train and is probed to provide a prediction of the duration of the trip (i.e., $t_{predicted}$) at at a frequency of once per minute.  Further, assume that Pat generates an invariant value for $t_{predicted}$ of 50 minutes whenever probed.  In this scenario, a conclusion of the rational model is that Pat's prior must decrease as $t_{past}$ approaches $t_{predicted}$. This scenario is illustrated in the top panel of Figure \ref{fig:PriorShift2Panel}.  Put simply, by the rational prediction model, Pat's prior must decrease rapidly when Pat's prediction is invariant.  Figure \ref{fig:PriorShift1} shows the time evolution of the prior under this scenario, something that is recoverable if we know both $t_{past}$ and $t_{predicted}$. What we see here is that to maintain an invariant $t_{predicted}$ as $t_{past}$ approaches it, the prior distribution shifts rapidly to the left.   

This reduction in prior, what we call the \textit{prior crash}, is paradoxical not because priors need to be fixed but because the change in prior is self-organized, without information external to the model.  Rapid changes in the prior are typically assumed to be driven by changes in the environment (see \cite{Sohn2021,Prat2021} for examples).

Another way to understand this paradoxical feature of the rational model is to consider another limiting case. When the prior is fixed (shown in the lower panel of Figure \ref{fig:PriorShift2Panel}), the prediction $t_{predicted}$ increases as $t_{past}$ approaches it.  For this case, we imagine that Pat has an invariant prior the effect of which is that Pat's prediction of when the train trip will end, $t_{predicted}$, increases as $t_{past}$ approaches it.  This is similar to asking Pat a different but common question. Imagine we ask Pat to predict a person's life-span; as $t_{past}$ (the current age of a person) approaches the expected value of the prior, the predicted life span increases--e.g., given that someone is 79 years of age (the expected value in the US \cite{Arias2019}), Pat's prediction, given the rational decision model, is a bit greater than 79.  

Taking both cases together, the invariant prediction and the invariant prior are just flip-sides of the same decision model. The first case forces the prior downward to maintain a constant prediction; the second case forces the prediction upward to maintained an invariant prior. 

However paradoxical the prior crash seems in the abstract, this feature of the rational decision model is an issue only to the degree that its paradoxical behavior is observed, something that is not guaranteed. As $t_{past}$ approaches $t_{predicted}$, it is possible that, in real-time contexts, humans adapt their predictions, either in an effort to minimize the degree of change in the prior or some other motive (we leave the motive for the discussion). 

Thus, the first objective of our work was to \textit{observe how participants adapt their predictions in real-time and to understand the relation of the  un-observable theoretic component, the prior, to the human predictions}.  The panels of Figure \ref{fig:PriorShift2Panel} provide some bounds on the range of what we expect to see. It may be that the boundary cases are not found; in particular, if the prior crash, as we call it, is not observed, e.g., due to adaptations in the human predictions, then the paradox implied by the rational model might not be an issue in practical terms. To colloquailize, maybe people just don't do that.  Our naturalistic observational context is well suited to this objective because the relation between what people predict $t_{predicted}$ and $t_{past}$ are unconstrained.

The second issue raised by our observational context focuses on the role of information external to the rational decision model.  The participants in the forecasting tournament were not limited in terms of the kinds of information used to make predictions.  An obvious candidate was the COVID-19 daily-case count reported in the State of Virginia (the forcasting tournament was focused on Virginia). In fact, this data source was provided to participants as a potential source of information in the context of the forecasting question (see \cite{VDHOnline} for the current daily-case count data in Virginia). 

Thus, our second objective was \textit{to explore possible relations between external data and the human predictions}. If the human predictions are correlated in some manner with the epidemiological case-counts, then it suggests a possible re-consideration of the rational model described above, depending on the nature of the relation.  A priori, it is not obvious how to integrate such information into the rational model as it is specified above.

Our methodology was descriptive but in a manner that included the theoretical unobservable constructs of the Bayesian decision model.  We observed participants predictions from the forecasting tournament in real-time, both in aggregate and as individuals, and recovered the priors implied by the rational decision model.  This afforded a view of the dynamics of human predictions and their respective priors over the course of the tournament.  In particular, we observed conditions such that $t_{past}$ approached $t_{predicted}$ and thus served our first objective well.  Further, to serve our second objective, we provide an analysis of the relation between the human predictions and the Virginia daily-case count data of the COVID-19 Omicron wave.  This analysis relied on the rational decision model in the sense that it was the theoretical model that suggested to us a useful measure to extract from the tournament data: prediction in the form of $t_{predicted}$.  We mention this point to emphasize a consistency across our two objectives that derives from the rational decision model. 

\begin{figure}
    \centering
    \includegraphics[width=0.75\textwidth]{Figures/Theory_PriorShift_1_2panel.png}
    \caption{\footnotesize{The theoretical relation between rational priors, $t_{past}$ and $t_{predicted}$ given the rational decision model for a hypothetical human riding on a train.  The x-axis represent the time at which a prediction is made, $t_{past}$;  the black, solid line represents the human decision, $t_{predicted}$; the black, large dashed line shows the prior, given $t_{past}$ and $t_{predicted}$, generated by the rational decision model; the human horizon reflects the amount of time until the predicted event from $t_{past}$ (i.e., its how far forward is the prediction from the moment it is given).  The top panel illustrates the invariant human decision case (a fixed $t_{predicted}$); as $t_{past}$ approaches $t_{predicted}$, the prior necessarily drops (what we call the "prior crash").  The lower panel shows the invariant prior case (a fixed prior distribution); as $t_{past}$ approaches $t_{predicted}$, $t_{predicted}$ increases.  
    }}
    \label{fig:PriorShift2Panel}
\end{figure}

\begin{figure}
    \centering
    \includegraphics[width=0.75\textwidth]{Figures/Theory_PriorShift_1.png}
    \caption{The relation between prior and posterior distributions given three instances of the rational decision model; these instances were derived from the scenario shown in the top panel of Figure \ref{fig:PriorShift2Panel} for the condition of an invariant human decision, $t_{predicted}$, while riding on a train.  In each panel, the black solid line shows the posterior distribution; the black large dashed line shows the posterior distribution; the vertical grey small dashed line annotates the fixed invariant decision of 50 minutes.  Moving from the top to the bottom panel, this figure shows the leftward shift in the prior as $t_{past}$ increases from 41 to 47 to 49 minutes.  During this shift the median of the prior distribution shifts from 50 to 44 to 32 minutes while the median of the posterior distribution is maintained at 50 minutes.
    }
    \label{fig:PriorShift1}
\end{figure}

%Figure \ref{fig:TheoryHeatMap}   
%\begin{figure}
%    \centering
%    \includegraphics[width=\linewidth]{Figures/TheoryHeatMap_1.png}
%    \caption{The x-axis shows $t_i$; the y-axis shows the median of the prior used in a decision (is the same as $t_{total}$).  The heatmap values were computed as the median of the posterior given a prior distribution and $t_i$ minus the median of the prior.  The key point of this graphic is depicted by the yellow and blue bands:  this shows that as $t_i$ approaches the median of the prior, the estimate of $t_{total}$ increases. To read this graphic, pick a value for median of prior (on the y-axis) and follow it from left-to-right (to simulate increasing $t_i$). 
%    }
%    \label{fig:TheoryHeatMap}
%\end{figure}


\section{Methods}\label{methods}
We present our work in two studies.  Study 1 provides a descriptive overview of the human predictions during the course of three rounds in the forecasting tournament, both in aggregate and idiographically (for some individuals).  This will provide a sense of the data structure and the performance of the forecasters.  The primary focus for Study 1, however, was to tackle our first objective: observe how participants adapt their predictions in real-time and to understand the relation between the unobservable theoretic component, the prior, and the human predictions.  Study 2 addressed our second objective: to explore possible relations between external data, the human predictions and, potentially, to components of the rational model (e.g., the prior).  To this end, we used the same human predictions from Study 1 and provided an analysis of the relation of the human predictions to the epidemiological case-count in Virginia. Next, we provide general methodological details that apply to both studies prior to detailing the specific methods used in each study.

\subsection{Forecasting Interface}\label{interface}
Participants were not restricted in terms of the number of forecasts they could make, given they were within the forecasting window (the window of time for which forecasters were permitted to submit predictions).  The interface presented to participants the prediction window of the question (the window of time for which predictions were made) as a timeline; overlaying the timeline was a visual representation of a probability density distribution (logistic in form).  The participant entered a forecast by manipulation of three horizontal sliders, two of which adjusted the width (left, right) of the probability distribution and one of which adjusted its center.  Further, the participants could generate a mixture distribution for their prediction as a layered and weighted combination of up to five separate distributions.  
%MUST GET ANSWER ON NON-SYMMETRY OF HAND GENERATED FUNTION VS LOGISTIC.
%Formally, this is:
%\begin{equation}
%    f(x) = \sum_{k=1}^{5} \pi_k g(x\vert \mu_k,s_k)
%\end{equation}
%where $\pi$ represents weights for each of the five (up to five) distributions.  Each distribution was defined as:
%\begin{equation}
%    g(x\vert \mu_k,s_k) = \frac {e^{-x(x-\mu)/s}} {s(1+e^{-x(x-\mu)/s})^2}
%\end{equation}
%The \pi_i were constrained so that their sum was $\le$ 1.
In addition to the visual of the prediction distribution, the participant saw the median, and quartiles of the distribution both visually (as vertical lines in the probability distribution) and numerically.  The questions under investigation in this article were discrete in nature (day); the numerical value of the prediction was presented at the day resolution to the participant as a summary measure of the participants probability distribution.  

\subsection{Data Structure and Major Constructs}\label{data-structure}
These data were extracted from the first three rounds of an online prediction tournament run by Metaculus and using the Metaculus forecasters pool as the participants (we describe the pool in section Participants).  The dates available for forecasting (the forecasting window) per rounds 1-3 were, respectively: 11-12-2021 to 12-03-2021; 12-03-2021 to 12-24-2021; 12-24-2021 to 01-14-2022.  The prediction window (the horizon of the forecast) for the forecasts per rounds 1-3 were, respectively: 11-12-2021 to 02-03-2022; 12-03-2021 to 02-25-2022; 12-24-2022 to 03-18-2022.  Each forecast made by a participant was in answer to the following question:  "When will the peak .. occur ". For the remained of this article, we call this the duration of the epidemic as short-hand for the duration from start to the peak of the epidemic curve. 

To generate $t_{past}$ we defined $t_0$ to be the beginning of the Omicron wave of COVID-19 and set it to November 29, 2021; this date was based on a reasonable estimate considering the daily reported case-count curve in Virginia (see \cite{VDHOnline}).  To construct $t_{predicted}$ for each forecast, we first transformed the participant input to the forecasting interface (described above) into a single date value that approximated the observed median of a prediction.  The data structure provided by the forecasting interface was in the form of a discrete probability mass function of ordered time intervals over an 84 day prediction window.  This generated 101 evenly spaced time intervals of the length of 20 hours, 9 minutes and 36 seconds (this interval times 101 = 84 days) each with a probability.  We defined the median of the prediction to be the discrete value $X$ for which $X \le 0.50$ when transforming the discrete probability mass function into a cumulative probability mass function.  In short, this procedure defined a single value, the approximate median value, as predicted date. Then, to construct $t_{predicted}$, we subtracted $t_0$ from the predicted date (e.g., Jan 2, 2022 minus November 29, 2021 $= t_{predicted}$ of 34 days).  Further, another useful construct is the prediction horizon, what we call the \textit{human horizon}; we defined this to be the number of days between the date on which a prediction was generated and the predicted date of the prediction.  This measure was useful because it represents the distance between $t_{past}$ and $t_{predicted}$ directly, a measure that theoretically important in terms of the potential for the rational prediction model to generates seemingly paradoxical behavior, as described in the Introduction.  In particular, in our analysis we were concerned with behavior of the rational model when the horizon was small.  We also extracted the date the prediction was generated by the participant, a measure that did not require transformation.

In short, the forecasting tournament data provided the date the prediction was generated by the participant (e.g., December 20, 2021) and the predicted date of the epidemic peak (e.g., Jan 2, 2022). Given a pre-defined beginning of the epidemic curve, $t_0$, we could generate both $t_{past}$ and $t_{predicted}$ as required by our theoretical model.  Further, we use the human horizon to understand the relation between the human predictions and the rational model when the human horizon is small.

The other major data structure reflected a measurement of the daily-case count for COVID-19 in Virginia.  These publicly available data were provided by the Virginia Electronic Disease Surveillance System (VEDSS) \citep{VDHOnline} and are entered by 5:00 PM the prior day (but are subject to change as quality assurance was ongoing).  VEDSS adopted the CDC COVID-19 2021 Surveillance Case Definition \citep{CDCguide} on September 1, 2021.  The quality assurance methods for VEDSS in respect to these data can be accessed here \citep{VDHquality}.

The raw form of these data is the cumulative number of cases on each day.  We transformed this into the daily number of cases per day by computing daily differences.  We then computed the rolling seven-day average over these raw data.  Because we had data for days well beyond both end-points of interest (11-30-2021 and 01-14-2022), our data structure on those days reflects the rolling seven day average prior to and after those end-points where appropriate.

\subsection{Participants}\label{participants}
All participants were users on the Metaculus forecasting platform (see \citep{metaculus}).  Metaculus is an online platform that hosts a community of forecasters (N is approximately XXXX). It is publicly accessible (with free registration for an active account) and hosts individual questions and tournaments with monetary prizes.  The Metaculus platform does not collect demographic data on the forecasters.  However, for the participants reported in this article, we have limited summary information on the distribution of competence from prior forecasts. [WAITING ON METACULUS TO SEE IF THIS IS AVAILABLE FOR THE SET 39 Forecasters in our analysis]

\subsection{Pre-processing of Forecasts}
The design of the forecasting tournament necessitated some pre-processing of these data; we selected/filtered based on criteria applied to the predictions not to participants. We started with 471 predictions that were generated by 47 participants.  The first step was to filter out predictions that were using the end-points of a round's prediction window, what we dub end-point forecasts; left end-point forecasts were at the earliest point of the prediction window; right end-point forecasts were at the latest point.  Because the prediction windows were bounded, we could not know if a end-point forecast was for the endpoint or beyond it.  None of the predictions fit this criteria.  The second step was to filter out the predictions that were prior to our definition of the start of the epidemic wave, $t_0$.  This reduced the number of predictions to 403 and the number of participants to 39.  The final filtering process was to remove predictions that were less than $t_{past}$ because these are theoretically impossible (impossible by the theory).  This removed two predictions for a final usable total of 401 predictions generated by 39 participants.  The distribution of the number of forecasts per user was highly right-skewed (mean=10.33, std=15.98, median=3, skew=2.15).

\subsection{Methods: Study 1}
\subsubsection{Theoretical Model}
The primary use of the theoretical model in Study 1 was to recover the best prior given a participant's prediction, $t_{predicted}$, and the value of $t_{past}$ at the time of the participant's prediction.  Our theoretical model, described in the Introduction, was borrowed directly from \citep{GriffithsTenenbaum2006} and differed only in its implementation. We used a discrete, sampled prior and numerically computed the prior probability mass distribution from it; the likelihood was also computed from the same probability mass distribution.  Specifically, we used the Poisson distribution for the priors with the following probability mass function: 
\begin{equation}\label{eq:poisson}
   P(X=k) =  \frac {\lambda^k e^-{\lambda}}{k!.}
\end{equation}
where $k$ was the number of days and $\lambda$ was the expected number of days in an epidemic wave.  For each value of $\lambda$, across the range from 20 to 200 days, we generated 1000 samples numerically, using the scientific software 'Numpy'\footnote{We decided to use numerical samples in place of exact results to allow for a simple method for integrating sample biases into future modeling efforts or for integrating empirical estimates of real infectious disease wave durations.} from which we could compute the sampled probability mass function over the values of $k$ that were sampled numerically for a given $\lambda$.  This had the effect of restricting the prior and likelihood to the sampled values, but we desired a simple, approximate value for this study.

To find the prior distribution that was the best candidate for generating the human prediction $t_{predicted}$, we generated a three-way table that represented all combinations of prior distributions (medians of), $t_{past}$, and the median of the Bayesian decision model output $t_{predicted}$\footnote{Note that $t_{predicted}$ can refer to the human decision or the output of Bayesian decision model; this term will be used for both interchangeably} for the values constrained by the expected values $\lambda$ of our priors (20 to 200 days).  To recover the best candidate prior for a human prediction, we searched the three-way table for the closest match--i.e., we found the prior distribution that would have generated $t_{predicted}$ given $t_{past}$.  For purposes of analysis, we reported the expected value of the prior's distribution as the measure for analysis. Henceforth, we call this measure the prior; the context of our use of the term prior should help the reader to disambiguate the meaning of the expected value of the distribution from the distribution itself.  

By design, this method does not restrict a single prior to an individual over the course of the tournament, but only to a single prediction within the tournament; we wanted to see the extent to which an individual's prior varied over the course of the tournament, if at all.  

\subsection{Methods: Study 2}
This study used two data structures: (i) all 401 human predictions in terms of the predicted duration of the epidemic, $t_{predicted}$; (ii) the daily-case count for COVID-19 in Virginia, henceforth called case-count data.  

The human predictions were averaged by day (all predictions in a 24 hour period were averaged together) and then fitted using linear polynomial regression (where datetime was transformed to a linear variable as the predictor; we do not provide the coefficients here, but will report them upon request or they can be recovered by using our publicly available source code: \cite{orrgit}).  The purpose was to provide a relatively low-dimensional and smooth representation of the temporal signature of these data for our next processing step. Using the polynomial solution, we then computed, numerically, its first and second differentials.  We used the same method for the case-count data. 

In addition, we used a change-point method for the case-count data, based on the sequential discounting auto-regression learning (SDAR) algorithm \citep{Yamanishi2002}, to detect change points in the case-count data as we transformed it, prior to the polynomial fit, using a discount rate of 0.01, order of 3 and a smoothing factor of 5 days. 

\section{Results}\label{results}
\subsection{Results: Study 1}
The three key measures for this study were the human predictions $t_{predicted}$, the human horizon and the theoretical prior (as described in the Methods section, we are reporting the expected value of the recovered prior distribution).  Figure \ref{fig:All_Ss_t_tot} shows the aggregated result for Study 1.  We present these aggregated results in part to orient the reader to the nature of these data and the method of our descriptive analysis.  Keep in mind the following when referring to Figure \ref{fig:All_Ss_t_tot}:  (i) we use the date of prediction, which is constrained to be within the forecasting window, not the prediction window for the x-axis (see Methods for details of these measures); (ii) given (i), $t_{predicted}$ is the predicted total duration of the epidemic for a specific date, (iii) ground truth horizon was provided as a point of reference; it reflects what would be an accurate horizon given the actual epidemic peak of January 13th, 2022 and is directly comparable the human horizon in its meaning.  

The general features of this figure are as follows: (i) the human predictions of the duration of the epidemic, $t_{predicted}$, systematically increase to about half-way through the forecasting window, followed by a decline; the final portion of which shows an increase; (ii) most of the forecasting period shows a human horizon somewhat greater than the ground truth horizon, except for the beginning of the forecasting window and somewhat near the end of the forecasting window(around January 7th, 2022); (iii) the human horizon shows an upward trend during the end of the forecasting window.

\begin{figure}
    \centering
    \includegraphics[width=\linewidth]{Figures/Good_all_s.png}
    \caption{The two primary measures, human $t_{predicted}$ and the rational prior (using the median of the recovered prior), over the course of the tournament. (The y-axis shows days; the x-axis shows date of human prediction.)  The human horizon reflects human $t_{predicted}$ minus the date the prediction was generated by the participant.  The ground truth horizon shows the number of days until the actual epidemic peak in Virginia (January 13th, 2022) from each date in the figure; it is the ground truth equivalent of the human horizon.  All trend lines (except the groud truth horizon) were computed as 4-day moving averages. 
    }
    \label{fig:All_Ss_t_tot}
\end{figure}

In terms of the objective of Study 1--to understand the relation between $t_predicted$ and the prior, with emphasis on predictions that have a small horizon--we see in Figure \ref{fig:All_Ss_t_tot} some signature of the predicted \textit{prior crash}, but in attenuated form.  In short, the observations support, provisionally, some middle ground between the two limiting cases that were described in the Introduction (the constant prior vs the constant decision).  The upward trend of the human horizon toward the end of the forecasting window is suggestive of an adaptation on the part of the forecasters as the horizon closes.  

This provisional conclusion is better understood when considering the non-aggregated data (showing all 401 predictions) as shown in Figure \ref{fig:All_S_Scatters}.  In the top-left panel of this figure, we see that the relation between the human predictions and the rational prior are close in value for many of the predictions with a subset of predictions that show larger values for $t_{predicted}$ compared to the prior.  The top-right panel of Figure \ref{fig:All_S_Scatters} shows this result over time, a plot that disaggregates the aggregate figure (Figure \ref{fig:All_Ss_t_tot}) and shows that the difference between a prediction and its prior only manifests near the end of the forecasting window.

The lower two panels in Figure \ref{fig:All_S_Scatters} are the most telling, in terms of the theoretical predictions.  The lower-left figure shows the difference between the human prediction and the prior as a function of the human horizon; it clearly shows that any substantial difference between the two only manifests with human horizons less than 5 days; the lower-right panel shows this result over the forecasting window.  

In summary, Figures \ref{fig:All_Ss_t_tot} and \ref{fig:All_S_Scatters} are suggestive of some adaptation on the part of the participants when the human horizon is small, a suggestion that is more strongly supported by the individual-level data from the most active participants.  We review these data next.

\begin{figure}
    \centering
    \includegraphics[width=\linewidth]{Figures/Study1_ExplanatoryScatter_All_S.png}
    \caption{The top-left panel shows the relation between the human predictions and the rational prior; The top-right panel shows this relation over time. The lower-left panel shows the difference between the human prediction and the prior as a function of the human horizon; the lower-right panel shows this result over the forecasting window.}
    \label{fig:All_S_Scatters}
\end{figure}

Figure \ref{fig:Single_Ss_t_tot} shows a parallel result to Figure \ref{fig:All_Ss_t_tot} but for the most active subjects (those with more than 15 predictions).  All of the most active participants show some separation of the prior from the human predicted duration of the epidemic, something that is illustrated to a larger degree for participants numbers one, three, five and six.  Further, all of these participants (excepting participant number 4) show the following two signatures of adaptation: an upturn of the predicted duration of the epidemic accompanied by a flattening of the human horizon.  Further analysis of the disaggregated data is shown in Figure \ref{fig:Single_Ss_Summary_2}, a replicate of the lower-right panel of Figure \ref{fig:All_S_Scatters}. For this figure, we added lines that represent the time course for three of the six participants to illustrate the temporal peak of the largest differences between the predicted duration of the epidemic and the prior.  There was a clear peak prior to the end of the prediction window (these are all small horizon predictions, as gathered in the lower-left panel of Figure \ref{fig:All_S_Scatters}) followed by a set of predictions that had a reduced difference between the predicted duration and the prior, likely driven by an increase in the value of $t_{predicted}$.  It is important to note that for each of these three participants, there were multiple predictions both falling within the peak window and post-peak. 

The only participant with large differences between the predicted duration of the epidemic and the prior whom did not show similar behavior was number two; however, this amounts to only only one prediction that peaked for this participant which came very late in the forecasting window.  It is plausible that participant number two would have exhibited a similar signature as the other three participants given time.  The remaining two participants (numbers three and four)  did not exhibit such large differences between their predictions and their recovered priors. 

\begin{figure}
    \centering
    \includegraphics[width=\linewidth]{Figures/Single_Subjects.png}
    \caption{
    This figure depicts the same three components as in Figure \ref{fig:All_Ss_t_tot} for the six subjects with the highest response rate ( $> 15$ responses). Participant number is provided in the top-left corner of each panel.
    }
    \label{fig:Single_Ss_t_tot}
\end{figure}

%%%%%%%
%I DON"T THINK WE NEED THIS FIGURE IT JUST REPLICATES WHAT WE SEE
%WITH ALL Ss
%\begin{figure}
%    \centering
%    \includegraphics[width=\linewidth]{Figures/Single_Ss_Horiz_Summary.png}
%    \caption{
%    This figure 
%    }
%    \label{fig:Single_Ss_Summary_1}
%\end{figure}

\begin{figure}
    \centering
    \includegraphics[width=\linewidth]{Figures/Single_Ss_Date_Summary.png}
    \caption{This figure is identical in form to the lower-right panel in \ref{fig:All_S_Scatters}; it shows the six most active participants, each labeled in the figure legend ("P" stands for participant).  The colored, dashed-lines show individual participants (green, participant six; red, participant five; black, participant one). The solid, vertical grey line falls on January 9th, 2022, the date at which (on and after) the predictions for participants one, five and six were post-peak.  See the text for details.}
    \label{fig:Single_Ss_Summary_2}
\end{figure}

\subsection{Results: Study 1 Discussion}
The primary purpose of Study 1 was to understand, in a naturalistic, observational setting, the relation between human predictions of duration and the unobserved prior, given the Bayesian decision model we described in the Introduction.  A principle value of the observational context was that $t_{predicted}$ and $t_{past}$ were not mutually constraining.  We hoped to and did observe cases for which $t_{past}$ approached $t_{predicted}$--the interesting case in terms of the Bayesian prediction model and its relation to human predictions. 

We came to this study without precise predictions, but with bounds on what we expected to see as presented in Figure \ref{fig:PriorShift2Panel}: the invariant prediction and the invariant prior.  Neither was observed.  Instead, our observations were consistent with the notion that the participants were adapting their predictions as the difference between the predicted duration of the epidemic and the prior deviated.  We saw signatures of this in both the aggregate time-series and in the individual-level time series.  The most suggestive result was that presented in Figure \ref{fig:Single_Ss_Summary_2} where we see a clear reduction in the difference between predicted duration and the prior after they deviated a large degree. 

Study 1 presupposes that the Bayesian prediction model operates without knowledge of the difference between $t_{predicted}$ and the expected value of the prior; this kind of comparison is not computed in the model.  So, by claiming that we have provisional evidence for human adaptation, we are also claiming, strictly speaking, that the decision model computes something in relation to this difference.  We leave this for future experimental study in which proper control would be instrumented.

One final comment for discussion.  It is paramount to understand that there is no necessity for the prior to split from the predicted value; the participant can always increase their predicted duration to a value large enough so that the expected value of the prior matches it.  In short, pulling up on the predicted value always works, if you pull hard enough.  We did not see this in these data, but a less severe adaptation.

\subsection{Results: Study 2}
The key measures for Study 2 were the case-count data (after our transformation) and the set of human predictions of the duration of the epidemic ($t_{predicted}$).  Figure \ref{fig:Study_2_Multi} shows the results.  In the left panel, we see the case-count data and the human predictions super-imposed, along with the polynomial fits.  The three vertical lines indicate the change points in the case-count data.  The first noticeable feature in this panel is the large region of time after the second change point for which the case-counts and the human prediction move in opposing directions.  Another interesting feature is that the case-count curve was much flatter prior to the second change point compared to after it.  The final point of interest in this panel the similar slope of the two time-series in terms shortly prior to and after the third change point.  The second panel, showing the first derivatives of the case-count polynomials, provides a more nuanced understanding of the relation between the human predictions and the case-counts.  The major features of this panel are as follows:  (i) the strongest and most sustained growth of the case-counts is the period for which we see the strongest and most sustained negative growth of the human predicted durations; (ii) during the period mentioned in (i) we don't see positive growth in the human prediction until the case-count growth starts to slow.  In short, from the derivatives, we see two clear relations.  First, when the case-count growth is strong, the human growth is negative.  Second, when the case-count growth is slow or lessening from a high growth period, the human predictions show growth. 
 
\begin{figure}
    \centering
    \includegraphics[width=\linewidth]{Figures/Study_2_Figure.png}
    \caption{The left panel illustrates the case-count data and the human predictions super-imposed, along with the polynomial fits.  The three vertical lines indicate the change points in the case-count data.  The right panel shows the first derivatives of the polynomials for the case-count and human predicted durations.}
    \label{fig:Study_2_Multi}
\end{figure}

\subsection{Results: Study 2 Discussion}
The purpose of Study 2 was to provide an analysis of the potential relation between external data (in the form of epidemiological daily case-counts) and the human predictions of the duration of the epidemic Omicron wave.  We did not have any predictions coming into this study in large part because the Bayesian decision model, as we specified it, does not afford the integration of data of this kind (or of any external information). Our discussion, therefore, will begin without consideration of this model. 

The relation between the human predictions and the case-counts, by some measure, was rather stark.  The human predictions seemed to respond to the growth of the case-counts in a way that was suggestive of how infectious disease operates.  In the beginning of an epidemic curve, slow growth represents a good deal of uncertainty: the epidemic might not take-off at all or it may be delayed significantly. Once an infectious disease agent shows strong growth, this can signal the potential for the infectious agent to rapidly burn through the susceptibles in the population, thus peaking sooner than expected.  Further, in a period of fast growth, if the growth again slows, it may signify a delay in the process.  It is plausible that humans are sensitive to this notion when making predictions about the duration of the epidemic.  

\section{General Discussion}\label{discussion}
%%%PRACTICAL THEORY
\noindent
We open the discussion with a quote \citep{Lewin1943}:
\begin{displayquote}
... there is nothing as practical as a good theory. (p. 118).
\end{displayquote}
The rational theory proved extremely practical, in the scientific sense.  Given only impoverished naturalistic observational data, the theory afforded a mathematical decision modeling framework grounded in prior experimental work\citep{GriffithsTenenbaum2006,GriffithsTenenbaum2011}, a set of intriguing (if not paradoxical) predictions (used in Study 1) and a potential limitation (use of external information, Study 2).  

The application of the model to the naturalistic observational data returned the favor. First, it offers fresh testable hypotheses.  In Study 1, we put forth the notion that participants were sensitive to and reacted to the difference between the $t_{predicted}$ and the prior; there was some evidence for what seemed like a dynamic response once this difference became relatively large.  This amounts to a testable, first-order hypothesis: are people actually sensitive to this measure?  To speculate, it should be possible to develop an experimental protocol to measure this hypothetical phenomenon that leverages existing experimental procedures and measures, e.g., \citep{sussman2007role}.  In Study 2, we saw a striking relation between the human predictions and the case-counts, one that is suggestive of an understanding of the epidemiological process. Given said hypotheses, the model provides a second service--pointers for the construction of new theoretical processes.  The scope of this article does not warrant proposing a new or amended model to accommodate external information (Study 2) or to incorporate the difference between $t_{predicted}$ and the prior (Study 1).  However, the flexibility of the theoretical approach is very general as demonstrated by its ability to accommodate a variety of findings even within narrow domains, e.g., anchoring bias \citep{Lieder2018}. 
The rational theory also proved practical, in the practical sense.  This work represents a step towards improving the psychological underpinnings of the kinds of human judgements that are directly relevant for a variety of policy, administrative or operational needs.  The tournament from which we gathered our data was developed in conjunction with the Virginia Department of Health (VDH) to provide insights into future case numbers, vaccination rates, booster uptake and other key indicators for VDH COVID-19 operations.  Our work represents only a slice of the tournament, but it begins the process of providing a scientific evidence-base for the forecasts, something valued in near real-time emergency decision-making \citep{Galea2021}.  
%
In this policy space, improved human-machine collaborations (especially when theory of mind is desirable) may yeild benefit.  The    
%
Another potential application in the policy decision space is towards developing high-fidelity population models of infectious disease.  Our work could provide the basis for the decision logic used by synthetic agents in large, at-scale agent-based models of infectious disease dynamics, an effort for which our laboratory is highly-experience (University of Virginia).  Recent work has called for the integration of agents grounded in first-principles of cognitive science and psychology [RCP, BRIMS, ]; our work represents one approach to this end.




%%END MAIN

%%GENERAL SCRAPS


\bibliography{example.bib}

%\appendix
%\section{} keep this line commented out
%    \section{APPENIX STUFF}\label{appendix_basic_hop}
Hi I'm your Appendix

%%%%%
\section{Instructions on Use of Interface}
[NOTE: from https://www.metaculus.com/help/faq/#howscore ]
Here are the instructions provided to participants regarding how to enter a prediction.

"How do predictions and points work for numerical questions?
Binary questions resolve to either 'yes' or 'no', but numerical questions resolve to a number. Therefore, when you make a prediction on a numerical question, you need to specify a probability for each possible outcome (that is, assign a probability density function). There are an infinite number of such functions, but to make life a little bit easier we restrict predictions to a mixture of up to 5 logistic distributions. You can adjust the center, width and weight of each component of the mixture using the sliders on the question pages.
Here are the key things you need to know, with more details below.
	1.	You should put the center of your distribution at what you think is the most likely value, then adjust the width of the distribution so that you attribute better than even odds to the true number falling into your range.
	2.	Making your distribution wider or narrower reflects your confidence in the central value you've specified, and decides the stakes: a narrower distribution gives more points if your central value is right, but makes you lose more if it's wrong.
	3.	As for binary questions, points are awarded both for being right and for being more right than the community. The points "on the line" will change over time, and the score you are awarded is time averaged over the time during which the question is open.
	4.	Some numerical questions restrict the possible resolutions to lie within a certain range. If the resolution falls outside of that range, then the question resolves ambiguously and no one receives any points. Other questions allow open-ended ranges, and you can assign probability to an out-of-bounds resolution by moving your distribution to the edge of the range."









%%%%
%%SCRAP BELOW HERE
%%%%

%Figure \ref{fig:TheoryPosteriorDists} 

%\begin{figure}
%    \centering
%    \includegraphics[width=\linewidth]{Figures/TheoryPosteriorDists_1.png}
%    \caption{
%    This figure depicts stuff
%    }
%    \label{fig:TheoryPosteriorDists}
%\end{figure}



%Figure \ref{fig:PriorShift2} visualizes the prediction of the theory as $t_i$ approaches the median of a prior.  
%\begin{figure}
%    \centering
%    \includegraphics[width=\linewidth]{Figures/Theory_PriorShift_2.png}
%    \caption{hi
%    }
%    \label{fig:PriorShift2}
%\end{figure}

%Figure \ref{fig:PriorShift1} visualizes the prediction of the theory as $t_i$ approaches the median of a prior.  
%\begin{figure}
%    \centering
%    \includegraphics[width=\linewidth]{Figures/Theory_PriorShift_1.png}
%    \caption{hi
%    }
%    \label{fig:PriorShift1}
%\end{figure}

%\begin{equation}\label{posteriordist}
%    \sum_{i=1}^{max} P(t_{total_{i}} \vert t)
%\end{equation}

\end{document}
