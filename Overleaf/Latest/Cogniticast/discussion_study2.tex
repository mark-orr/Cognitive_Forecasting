The purpose of Study 2 was to provide an analysis of the potential relation between external data (in the form of epidemiological daily case-counts) and the human predictions of the duration of the epidemic Omicron wave.  We did not have any predictions coming into this study in large part because the Bayesian decision model, as we specified it, does not afford the integration of data of this kind (or of any external information). Our discussion, therefore, will begin without consideration of this model. 

The relation between the human predictions and the case-counts, by some measure, was rather stark.  The human predictions seemed to respond to the growth of the case-counts in a way that was suggestive of how infectious disease operates.  In the beginning of an epidemic curve, slow growth represents a good deal of uncertainty: the epidemic might not take-off at all or it may be delayed significantly. Once an infectious disease agent shows strong growth, this can signal the potential for the infectious agent to rapidly burn through the susceptibles in the population, thus peaking sooner than expected.  Further, in a period of fast growth, if the growth again slows, it may signify a delay in the process.  It is plausible that humans are sensitive to this notion when making predictions about the duration of the epidemic.
